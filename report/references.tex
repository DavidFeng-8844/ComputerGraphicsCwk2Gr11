\documentclass[10pt]{article}
\usepackage[utf8]{inputenc}
\usepackage[T1]{fontenc}
\usepackage{geometry}
\usepackage{xcolor}
\usepackage{hyperref}
\usepackage{booktabs}
\usepackage{longtable}
\usepackage{array}
% Carlito is a metric-compatible font for Calibri
\usepackage[sfdefault]{carlito}

% Minimize margins to fit content
\geometry{
    a4paper,
    margin=0.6in,
    top=0.6in,
    bottom=0.6in
}

\setlength{\parskip}{0.4em}
\setlength{\parindent}{0pt}

% Remove headers and footers (page numbers)
\pagestyle{empty}

\hypersetup{colorlinks=true,linkcolor=blue!60!black,urlcolor=blue!60!black}

\begin{document}

\section*{References}

Blinn, J.F. 1977. Models of light reflection for computer synthesized pictures. \textit{SIGGRAPH Computer Graphics}. \textbf{11}(2), pp.192–198.

de Vries, J. 2020. \textit{LearnOpenGL: Learn modern OpenGL graphics programming in a step-by-step fashion} [Online]. [Accessed 11 December 2025]. Available from: https://learnopengl.com.

Hughes, J.F., van Dam, A., McGuire, M., Sklar, D.F., Foley, J.D., Feiner, S.K. and Akeley, K. 2014. \textit{Computer graphics: principles and practice}. 3rd ed. Upper Saddle River, NJ: Addison-Wesley Professional.

Lengyel, E. 2012. \textit{Mathematics for 3D game programming and computer graphics}. 3rd ed. Boston: Course Technology PTR.

\vspace{1cm}

\section*{Appendix: Individual Contributions}

The following table outlines the individual contributions of each group member to the codebase and the technical report. All members contributed substantially to their assigned modules.

\begin{longtable}{@{} p{0.15\textwidth} p{0.82\textwidth} @{}}
\toprule
\textbf{Member} & \textbf{Contribution Details} \\
\midrule
\textbf{Siyu Yu} & \textbf{Task 1.1:} Implement fundamental 4×4 matrix operations including multiplication, rotation, translation, and perspective projection, validated through comprehensive unit tests. \newline
\textbf{Task 1.2:} Create a functional 3D renderer with perspective projection, first-person camera controls using keyboard and mouse input, and simplified directional lighting. \newline
\textbf{Task 1.3:} Add texture mapping capabilities by loading and applying orthophoto aerial imagery to the terrain mesh combined with lighting calculations. \newline
\textbf{Task 1.4:} Demonstrate geometry instancing by rendering two launchpad models at different sea locations using material colors and a separate shader program, showcasing efficient resource reuse in 3D graphics. \\
\midrule
\textbf{Yujie Feng} & \textbf{Task 1.5:} Developed a procedural, hierarchical space vehicle model using geometric primitives, implementing affine transformations and inverse-transpose normal matrix calculations for correct shading. \newline
\textbf{Task 1.6:} Implemented the Blinn-Phong reflection model for multiple point lights, incorporating physically-based inverse-square distance attenuation and dynamic uniform updates. \newline
\textbf{Task 1.7:} Created a physics-based procedural animation system with a parametric curved trajectory, utilizing analytic derivatives and Rodrigues' rotation formula for orientation alignment. \newline
\textbf{Task 1.8:} Designed an advanced camera control system with a state machine architecture, implementing automated "Follow" and "Ground" tracking modes with seamless state transitions. \\
\midrule
\textbf{Haoyu Zhu} & \textbf{Task 1.9:} Refactor the rendering logic into a unified function, implement left-right split-screen rendering, maintain two separate cameras with independent controls, and support split-screen toggling and window size adaptation. \newline
\textbf{Task 1.10:} Implement a particle system that uses a pre-allocated particle pool to emit particles from the engine, renders and handles depth in a specific way, and clarifies the implementation assumptions and limitations. \newline
\textbf{Task 1.11:} Build an immediate-mode UI system based on OpenGL + GLFW, include basic interactive components, specify the steps for adding new UI elements, and use a font atlas to optimize text rendering. \newline
\textbf{Task 1.12:} Implement GPU and CPU performance measurement functionality, analyze the test results, and identify performance bottlenecks and the system's real-time performance. \\
\bottomrule
\end{longtable}

\end{document}
