\documentclass[10pt]{article}
\usepackage[utf8]{inputenc}
\usepackage[T1]{fontenc}
\usepackage{geometry}
\usepackage{graphicx}
\usepackage{booktabs}
\usepackage{xcolor}
\usepackage{hyperref}
\usepackage{helvet}
\usepackage{inconsolata}

\geometry{margin=0.75in}
\renewcommand{\familydefault}{\sfdefault}
\setlength{\parskip}{0.4em}
\setlength{\parindent}{0pt}

\hypersetup{colorlinks=true,linkcolor=blue!60!black,urlcolor=blue!60!black}

\begin{document}

\begin{center}
\textbf{\large Task 1.5: Programmatic Space Vehicle Model}\\[0.3em]
\textit{COMP3811 Computer Graphics -- Coursework 2}
\end{center}

\textbf{Overview.}
This section documents the procedural construction of a 3D space vehicle composed of geometric primitives. The model incorporates eleven components from four primitive types (cylinder, cone, box, sphere), employs translation, rotation, and non-uniform scaling, and positions components hierarchically. The vehicle is placed on \textbf{Launchpad~A} at world coordinates $(75, -1, 20)$.

\textbf{Primitive Generation.}
Four mesh generators in \texttt{primitive\_shapes.cpp} produce the geometric primitives: \texttt{generate\_cylinder} creates closed cylinders with outward-facing normals; \texttt{generate\_cone} computes face normals via cross products; \texttt{generate\_box} produces cuboids with per-face normals; \texttt{generate\_sphere} uses latitude-longitude tessellation with normalised position vectors as normals. All generators output positions, normals, and indices in \texttt{SimpleObjMesh} format.

\textbf{Vehicle Composition.}
The vehicle comprises eleven interconnected parts generated by \texttt{generate\_space\_vehicle()}:

\begin{center}
\small
\begin{tabular}{@{}llll@{}}
\toprule
\textbf{Component} & \textbf{Primitive} & \textbf{Dimensions} & \textbf{Colour} \\
\midrule
Main body & Cylinder & $r{=}0.8$, $h{=}8.0$ & Light grey \\
Nose cone & Cone & $r{=}0.6$, $h{=}3.0$ & Red-orange \\
Engine nozzle & Cylinder & $r{=}1.0$, $h{=}1.5$ & Dark grey \\
Fins ($\times$4) & Box & $0.15 {\times} 2.5 {\times} 1.0$ & Blue \\
Window & Sphere & $r{=}0.5$, $y$-scaled 0.6 & Cyan \\
Antenna & Cylinder & $r{=}0.1$, $h{=}1.5$ & Yellow \\
Thruster pods ($\times$2) & Cylinder & $r{=}0.3$, $h{=}1.5$ & Orange \\
\bottomrule
\end{tabular}
\end{center}

The main fuselage serves as the structural reference. The nose cone sits atop at $y{=}9.5$, demonstrating relative placement. Four fins are distributed radially at $90°$ intervals via composed rotation-translation matrices. The window applies non-uniform $y$-axis scaling (60\%) before translation. Thruster pods rotate $\pi/2$ about the $z$-axis before lateral positioning.

\textbf{Transformations and Normals.}
The \texttt{transform\_mesh} function applies $4{\times}4$ homogeneous matrices to vertex positions. Normals are transformed using the inverse-transpose matrix, preserving perpendicularity under non-uniform scaling. This ensures correct Blinn-Phong shading response across all components.

\textbf{Launchpad Placement.}
The vehicle is positioned on \textbf{Launchpad~A} (sea platform from Task~1.4) with vertical offset $y{=}0.2$ so the engine base rests on the platform surface. Position stored in \texttt{state.vehicleOriginalPos} for animation support.

\textbf{Screenshots.}

\textbf{[Figure 1.5.1: Front-oblique view of the space vehicle on Launchpad A]}

\textbf{[Figure 1.5.2: Elevated view showing radial fin arrangement and thruster pods]}

\end{document}
