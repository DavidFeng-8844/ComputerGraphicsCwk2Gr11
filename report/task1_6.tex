\documentclass[10pt]{article}
\usepackage[utf8]{inputenc}
\usepackage[T1]{fontenc}
\usepackage{geometry}
\usepackage{graphicx}
\usepackage{booktabs}
\usepackage{xcolor}
\usepackage{hyperref}
\usepackage{helvet}
\usepackage{inconsolata}
\usepackage{amsmath}

\geometry{margin=0.75in}
\renewcommand{\familydefault}{\sfdefault}
\setlength{\parskip}{0.4em}
\setlength{\parindent}{0pt}

\hypersetup{colorlinks=true,linkcolor=blue!60!black,urlcolor=blue!60!black}

\begin{document}

\begin{center}
\textbf{\large Task 1.6: Blinn-Phong Shading with Point Lights}\\[0.3em]
\textit{COMP3811 Computer Graphics -- Coursework 2}
\end{center}

\textbf{Overview.}
This section documents the implementation of the full Blinn-Phong shading model for point lights with standard $1/r^2$ distance attenuation. Three coloured point lights are positioned around the space vehicle, affecting all scene objects (terrain, launchpads, and vehicle). The directional light from Section~1.2 is preserved.

\textbf{Blinn-Phong Shading Model.}
The fragment shaders (\texttt{default.frag} and \texttt{material.frag}) implement the complete Blinn-Phong reflection model. For each point light, the lighting contribution is computed as:
\[
L = (k_d \cdot \max(N \cdot L, 0) + k_s \cdot \max(N \cdot H, 0)^{\alpha}) \cdot \text{attenuation}
\]
where $N$ is the surface normal, $L$ is the light direction, $H = \text{normalize}(L + V)$ is the half-vector between light and view directions, and $\alpha$ is the shininess exponent. The diffuse term $k_d$ provides base illumination, while the specular term $k_s$ creates highlights on glossy surfaces.

\textbf{Distance Attenuation.}
Point lights employ standard inverse-square distance attenuation:
\[
\text{attenuation} = \frac{1}{1 + d^2}
\]
where $d$ is the distance from the light source to the fragment. The constant term prevents division by zero when $d \approx 0$. This physically-based falloff ensures realistic light intensity decay with distance.

\textbf{Light Configuration.}
Three point lights with distinct colours are positioned around the space vehicle:

\begin{center}
\small
\begin{tabular}{@{}llll@{}}
\toprule
\textbf{Light} & \textbf{Offset from Vehicle} & \textbf{Colour (RGB)} & \textbf{Purpose} \\
\midrule
Point Light 1 & $(-3, 2, 0)$ & $(2.0, 0.6, 0.6)$ Red & Left side accent \\
Point Light 2 & $(+3, 2, 0)$ & $(0.6, 2.0, 0.6)$ Green & Right side accent \\
Point Light 3 & $(0, 3, -3)$ & $(0.6, 0.6, 2.0)$ Blue & Rear/top accent \\
\bottomrule
\end{tabular}
\end{center}

The lights are offset slightly from the geometry (2--3 units) to avoid numerical issues with the attenuation calculation and to provide visible coloured illumination on vehicle surfaces. Light positions are stored as offsets from \texttt{state.vehicleOriginalPos} and updated dynamically during animation (Task~1.7).

\textbf{Directional Light Preservation.}
The directional light from Section~1.2 remains active with direction $(0, 1, -1)$ normalised and colour $(1.0, 1.0, 1.0)$. This light uses the simplified ambient-plus-diffuse model (no specular) as specified in the original requirements, providing consistent base illumination across the entire scene independent of point light effects.

\textbf{Shader Implementation.}
The \texttt{blinnPhong()} function in both fragment shaders computes per-light contributions. Uniforms include camera position (\texttt{uCameraPosition}) for view vector calculation, light properties (\texttt{uPointLights[3]}), and material shininess (\texttt{uShininess}). The helper function \texttt{set\_lighting\_uniforms()} in \texttt{main.cpp} uploads all lighting parameters to both shader programs, ensuring consistent illumination across textured terrain and material-coloured objects.

\textbf{Global Application.}
All three point lights affect every rendered object: the terrain mesh (via \texttt{default.frag}), both launchpad instances, and the space vehicle (via \texttt{material.frag}). This unified lighting approach creates cohesive scene illumination with coloured highlights visible on nearby surfaces.

\textbf{Screenshot.}

\textbf{[Figure 1.6.1: Space vehicle and launchpad illuminated by three coloured point lights]}

\end{document}

